\chapter{Results}

This chapter will offer more details about the verification process of the Linux Home Automation Controller. \labelindexref{Section}{sec:testing} will present the tools and aspects considered while testing the functionality of the system. The second section will offer an evaluation of the results using a quantitative approach.

\section{Testing}
\label{sec:testing}

\subsubsection{Application and Filesystem Deployment to The Embedded System Board}
The application development and the generation of the Yocto Linux image were done on a x86 workstation. A mechanism was needed for rapidly deploying a root filesystem to the embedded system board without having to copy them to the SD card every time. The RouterStation Pro can be configured to boot an operating system binary from a network location, using Trivial File Transfer Protocol (TFTP) \abbrev{TFTP}{Trivial File Transfer Protocol}, so I set up the \texttt{tftpd-hpa} server to offer the \texttt{vmlinux} binary generated by the Yocto build process. 

The folder containing the root file system was exported on the workstation using \texttt{nfs-kernel-server} and the board was configured to mount it automatically by using a Linux boot parameter. The files with the Python source code were also placed in the exported folder, so any changes could be immediately tested on the embedded system.

\subsubsection{Testing the REST API and the Web User Interface}
Postman\footnote{\url{https://www.getpostman.com/}} was used to test the functioning of the REST API. It is a powerful testing suite that simplifies the building and execution of HTTP API requests and allows templates to be defined for the validation of the JSON object responses and status codes. Postman proved to be an invaluable tool in the development process of the REST API.

The Chrome Web Browser and Chrome Developer tools were used for testing the pages of the Web User Interface. They have features for inspecting the HTML DOM and CSS styles, debugging JavaScript Code, analyzing network performance and interacting with the web page using a console. After development, the functionality of the Web UI was also tested on other popular web browser, such as Firefox and Internet Explorer. It also works and offers a mobile-specific experience (thanks to Bootstrap) on the Android Operating System, using the Chrome app.

\subsubsection{Security of the System}

Without being intended as a main project objective or an architecture design goal, the system also manages to be secure and ensure that only authorised users (i.e. members of a family) are allowed to interact with it. Having multiple components that communicate with each other using different protocols, some aspects have to be considered. The connection between the central node and the wireless boards is secured using WPA2 for the WiFi hotspot and through a pairing mechanism for Bluetooth. Also, the REST API and the Web UI can only be accessed with a correct user and password.

\section{Evaluation}

Because the target platform for the Linux Home Automation Controller server is an embedded system, a project objective is low resource usage. This section will evaluate memory, CPU and filesystem usage. In the end, I will make a comparison between the energy consumption generated by Bluetooth and WiFi wireless modules on the sensor boards.


\subsubsection{Memory Usage}

Memory usage values of the server proces were obtained using the \texttt{procfs} special filesystem, from the \texttt{/proc/<pid>/status} file. They are show in \labelindexref{Listing}{lst:mem}. Notable values are \texttt{VmSize} - current virtual memory usage, \texttt{VmRSS} - resident set size, \texttt{VmData} - size of "data" segment. The RouterStation Pro has 128 MB RAM, but typically the entire system only uses ~56 MB, less than 50\% of the total, while the server process uses 16 MB. It is a good value, taking into account that this includes the Python interpreter and the Flask framework.

\lstset{language=make,caption=Server Process Memory usage values,label=lst:mem}
\begin{lstlisting}
VmPeak:    23736 kB
VmSize:    23672 kB
VmHWM:     16812 kB
VmRSS:     16812 kB
VmData:    11596 kB
VmStk:       152 kB
VmExe:         4 kB
VmLib:      8316 kB
\end{lstlisting}

\subsubsection{CPU Usage}

\fig[scale=0.9]{src/img/cpu.eps}{img:cpu}{Server Process CPU Usage}

CPU Usage was tested in a scenario with 3 devices connected to the system and data logging activated at an interval of 5 seconds. The results are shown in \labelindexref{Figure}{img:cpu}. It can be seen that the normal CPU utilisation is at 0\%. The 3 spikes in the graph correspond to 3 REST API requests for data measurement history. Usage of 20\% is acceptable because the application must retrieve the values from the database and build the JSON object with the response. The application is targeted to serve the inhabitans of a normal home, which means an average number of 3 persons. Supposing that they are using the application at the same time, a normal use case would involve not more than 2 REST API calls per second, which the server should be able to handle effortlessly.

\subsubsection{Filesystem Usage}

The \texttt{vmlinux} Linux kernel ELF file has a size of 9.2 MB. It fits on the 16 MB Flash memory of the RouterStation Pro, leaving the posibility for further functionality to be added, if the need arises. The 48 MB root filesystem is a small fraction of the 8GB SD card that it will be stored on. There is more than enough space left for the \texttt{sqlite} database file that will hold the measurement values.

I used \texttt{sqlite3_analyzer} to monitor the statistics of the database. A row in the \texttt{Data_values} table occupies an average of 34 bytes. Considering a storing interval of 10 seconds, one value measurement would generate the following amounts of data that need to be stored: 293.80 KB for a day, 2.05 MB for a week and 8.93 MB for a month.

\subsubsection{Wireless Boards Power Consumption} 

One other aspect to be considered is the power consumption of the wireless sensor boards, as they are supposed to be battery-powered. \labelindexref{Table}{table:reports} shows average current consumption values for the wireless communication modules of the boards, taken from the product datasheets. It is clear that the RN-171 and MRF24 WiFi modules draw the most current. Bluetooth is the more power efficient protocol, at the cost of bandwith and communication range. It is an acceptable trade-off for the current scenario, as high data rate is not needed. Finally, it is clear that the Bluetooth Low Energy standard used by RN4020 is more efficient than the older version of the RN42 module.

\begin{center}
\begin{table}[htb]
  \begin{tabular}{l*{6}{c}r}
    & Idle (mA) & Connected-Data Rx (mA) & Connected-Data Tx (mA) \\
    \hline
    RN-171 & 15 & 40 & 180 \\
    MRF24 & 10 & 85 & 154 \\
    RN42 &  5 & 25 & 45 \\
    RN4020 & 1.5 & 16 & 16 \\
  \end{tabular}
  \caption{Board current consumption in different states}
  \label{table:reports}  
\end{table}
\end{center}

As a conclusion, the implementation respects the low resource use and therefore low cost objective of the project. It is able to run on the RouterStation Pro embedded system board that I proposed. Finally, out of the two wireless communication protocols that were considered, Bluetooth is more suitable for the desired use case.